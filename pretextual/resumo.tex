%   Aqui deve ser inserido um resumo de 150 a 500 palavras (limitação de tamanho dada pela BU). A linguagem deve ser português e a hifenização já foi alterada. O resumo em português deve preceder o resumo em inglês, mesmo que o trabalho seja escrito em inglês. A BU também diz que deve ser usada a voz ativa e o discurso deve ser na 3ª pessoa. A estrutura do resumo pode ser similar a estrutura usada em artigos: Contexto -- Problema -- Estado da arte -- Solução proposta  -- Resultados.

% Atenção! a BU exige separação através de ponto (.). Ela recomanda de 3 a 5 keywords

\begin{resumo}[Resumo]
    Os sistemas de transporte se estabeleceram ao longo da história como um dos principais condicionantes ao desenvolvimento humano. Com o avanço das tecnologias computacionais, surgiram os Sistemas de Transporte Inteligentes (STI), nos quais sensores são empregados na infraestrutura de trânsito e seus participantes, de forma a gerar dados situacionais para auxiliar a tomada de decisão. Neste contexto, diversas tecnologias foram desenvolvidas, dentre intrusivas e não-intrusivas, ativas e passivas. Dentre as tecnologias não-intrusivas e passivas, existem as baseadas em vibração, realizadas através de sensores inerciais, as quais podem gerar percepção veicular de forma segura, não poluente e de baixo custo. Entretanto, ao contrário de áreas como a visão computacional, o sensoriamento inercial tem sido pouco explorado em STI, com as pesquisas correlatas trabalhando, em suma, com cenários controlados e foco na aplicabilidade de resultados, e não no desenvolvimento de uma solução confiável e aplicável a cenários do mundo real. Com diversas lacunas de pesquisa a serem exploradas, a principal delas refere-se à adaptabilidade da solução, uma vez que existem diversos fatores de dependência que interferem no referencial de forças analisado, dado a diversidade de cenários pelo qual um veículo pode trafegar. Sendo assim, a proposta de gerar percepção veicular de forma adaptativa leva em consideração três etapas, sendo elas a amostragem de dados, pré-processamento e processamento. Na primeira, foi desenvolvido uma metodologia para captação, desde posicionamento de sensores, sua colocação na estrutura veicular e referenciais de captação e análise. Na segunda, os dados brutos passam por pré-processamento a fim de melhorá-los, integrando-os às propriedades de dependência. Por fim, foram analisadas técnicas clássicas de Machine Learning e de Deep Learning, para realizar o reconhecimento dos padrões de percepção.
    
    % Atenção! a BU exige separação através de ponto (.). Ela recomanda de 3 a 5 keywords
  \vspace{\baselineskip} 
  \textbf{Palavras-chave:} 
  Percepção Veicular de Ambiente. 
  Propriocepção Veicular. 
  Sensoriamento Inercial. 
  Sistemas de Transporte Inteligentes.
  
\end{resumo}
