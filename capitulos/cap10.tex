\chapter{Conclusões e Discussões}
\label{cap:conclusoes_discussoes}

% ROAD SURFACE TYPE ARTICLE
% With the ITS development, some applications have become increasily present in daily life, such as ADAS and, soon, autonomous vehicles. However, these systems need many sources of situational data to perform their purpose well. Among the various information, the recognition of surface type is one of the most important, and it can be used in the most varied applications to support human or machine decision making. However, for a wide application, it is necessary to develop a recognition model that uses safe technologies, and that is able to understand different contexts through which a vehicle can travel.

% Based on the premise above, in this work we developed models for road surface type classification using inertial sensor signals, which by their passive approach are safe, non-polluting and generally low cost. Through the literature analysis, there are few related studies which were carried out with low accuracy values, without considering contextual variations. As a way to analyze and validate the model for application in real-world scenarios, we list the dependency factors that impact in the learning and generalization of the pattern recognition model, and aggregate them. Initially, we performed several data collections to obtain variations of these factors. After pre-processing these data, they were interpreted using Classical Machine Learning and Deep Learning techniques to identify which one was the most appropriate.

% Among the Classical Machine Learning techniques, we apply KMC, SVM and KNN that obtained low accuracy values in the validation, 60.42\%, 72.68\% and 74.79\%, respectively. In the application of these methods, we observed the difficulty of extracting high-level features to well represent the data classes, even when applied statistical methods which are widely adopted in the feature extraction of vibration-based data. In the application of Deep Learning approach, all techniques were able to extract complex features from the data, so in all the developed DNN models, accuracy values were greater than those obtained through the best developed model based on classical techniques. LSTM-based, CNN-based and CNN-LSTM-based models were analyzed, which obtained the best validation accuracy result of 92.73\%, 93.17\% and 92.77\%, respectively. In a general analysis, we consider the CNN-based model the best due to the f1-score and accuracy value. The model classified segments of dirt road with f1-score value of 90.85\%, cobblestone road with 85.84\% and asphalt road with 98.96\%. With these results, we notice that all the data classes were properly handled by the model, and it is not necessary to apply data class balancing. In this work, we did not compare our models with those related in the state-of-the-art due to the lack of detail on the methodology of related studies.

% With the public availability of the datasets that we produce, new studies can be carried out on the recognition and classification of the most diverse patterns in ITS. Related to the exteroception, can be recognized potholes, speed bumps, surface quality, among others. Related to proprioception,  can be recognized driving events such as acceleration, braking, turning right or left, as well as driving safety level. In relation to the road surface type classification, new researches can perform experiments with the permutation of datasets grouped by dependency factors, analyze the other data collection points, such as near and above the suspension, and on the dashboard, both present in the collected dataset. Deep Learning layers with different approaches can also be analyzed, such as the Gated Recurrent Unit (GRU) and Convolutional LSTM (ConvLSTM).

\section{Trabalhos Futuros}