\chapter{Conclusões e Discussões}
\label{cap:conclusoes_discussoes}

Os sistemas de transporte se estabeleceram ao longo da história como importante condicionante do desenvolvimento humano. Devido a sua vital importância surgiram os ITS para tornar as redes de transporte mais eficientes e seguras. As aplicações em ITS já estão presentes no cotidiano, tais como ADAS, e tendem a ser cada vez mais comuns em breve, com veículos autônomos. Contudo, para que estas aplicações operem de forma confiável, mostra-se necessário o emprego de tecnologias seguras na coleta de dados brutos; a diversificação e redundância destas fontes de dados; o desenvolvimento de modelos de IA que operem corretamente em contextos reais; e da geração de uma grande variedade de informações situacionais sobre o ambiente de trânsito e seus participantes. 

As diversas tecnologias para coleta de dados brutos empregados na geração de informações situacionais do modal de transporte terrestre podem ser classificadas entre abordagem passiva e ativa. Dentre as de abordagem passiva, os sensores inerciais se mostram uma tecnologia segura, não poluente, de baixo custo e de fácil instalação, sendo ideais para uso em larga escala em ITS. Com estes sensores, é possível produzir uma grande variedade de informações na forma de percepções veiculares, dentre exterocepções e propriocepções, tais como as três trabalhadas nesta pesquisa: tipo de superfície de pista, qualidade de superfície de pista e detecção de lombadas. Contudo, ao contrário de outras tecnologias, tais como câmera ou LIDAR, as soluções baseadas em sensores inerciais levantadas através da nossa RSL mostram-se sumariamente provas de conceito simplistas, que não consideram variações das condições contextuais nas quais a solução será submetida quando aplicada em cenários do mundo real. Desta forma, mostram-se inaptas a se adaptar a diferentes contextos e, portanto, não são confiáveis para sua ampla utilização. 

Baseado no problema supracitado, consideramos que a adaptabilidade é um fator essencial para prover a confiabilidade necessária que possibilite uma ampla aplicação de soluções com sensores inerciais em ITS. Sendo assim, neste estudo produzimos uma metodologia para desenvolvimento e validação de modelos de percepção veicular que operem de forma adaptativa, ou seja, que consigam generalizar seu aprendizado para contextos desconhecidos. Com este objetivo, identificamos inicialmente os fatores de dependência que afetam os dados dos sensores inerciais e, consequentemente, o processo de aprendizado e generalização dos modelos. Esses fatores foram agrupados em quatro propriedades, sendo elas sensoriais, veiculares, ambientais e de condução. Por meio do estabelecimento das propriedades de dependência, realizamos várias coletas de dados para obter variações desses fatores. As variações incluíram dados coletados em três diferentes colocações no veículo, em três veículos diferentes, dirigidos por três motoristas, trafegando em três tipos de superfície de pista, as quais apresentaram variações no estado de conservação, além da presença de obstáculos e anomalias.

Com os dados coletados, produzimos um \textit{design} experimental para validar nossa hipótese inicial, a qual considera que através de conjuntos de dados que bem representem a diversidade contextual envolvida na aplicação dos sensores inerciais em ITS para geração de percepção veicular, ou seja, que os dados contenham variações contextuais significativas em relação aos fatores de dependência da solução, é possível construir modelos de IA capazes de aprender as relações e a influências dos fatores de dependência nos sinais dos sensores, possibilitando a generalização de seu aprendizado para cenários desconhecidos de forma confiável. Além deste aspecto, nossos experimentos analisaram também a utilização de diferentes domínios de análise, características de entrada, tamanho da janela de dados, e o impacto dado o ponto de coleta no veículo. Adicionalmente, analisamos a aplicação de técnicas com diferentes abordagens, dentre técnicas de Aprendizado de Máquina clássico e \textit{Deep Learning}, para validar a mais apropriada.

Em uma análise macro, as técnicas baseadas em \textit{Deep Learning} se mostraram consideravelmente superiores quando comparadas as técnicas clássicas, evidenciando a capacidade desta abordagem de extrair características de alto nível diretamente dos dados brutos que bem representassem o problema tratado, e seu excelente aprendizado em camadas com base nesses parâmetros. Todos os melhores modelos resultantes são baseados em \textit{Deep Learning}. Para classificação de tipo de superfície entre terra, paralelepípedo e asfalto, a rede CNN modelada classificou os segmentos com 93,04\% de acurácia em validação para os dados de próximo e abaixo da suspensão; 92,02\% para próximo e acima da suspensão; e 93,05\% para o painel de controle. Para classificação de qualidade de superfície de pista entre boa, regular ou ruim, a rede CNN modelada classificou os segmentos com acurácia em validação de 93,62\% para os dados de próximo e abaixo da suspensão; 93,90\% para próximo e acima da suspensão; e de 93,04\% para o painel de controle. Por fim, no reconhecimento de lombadas, a rede híbrida CNN-LSTM identificou o obstáculo com acurácia em validação de 97,97\% para os dados de próximo e abaixo da suspensão; 98,94\% para próximo e acima da suspensão; e 98,87\% para o painel de controle. Os valores de acurácia correspondem a média obtida entre experimentos por variação de contexto, onde o modelo treinado foi submetido a um veículo, motorista ou ambiente desconhecido. Com base nos resultados, podemos observar que os modelos conseguiram apreender e generalizar corretamente as percepções independente do ponto de coleta no veículo ou do desconhecimento de características contextuais. Sendo assim, a hipótese deste estudo é validada, de forma que os modelos se mostraram confiáveis para generalizar seu aprendizado para contextos desconhecidos, diferentes daqueles nos quais o aprendizado ocorreu. 

\section{Trabalhos Futuros}

Com a disponibilização de todo o material deste estudo, como conjuntos de dados, códigos-fonte e modelos treinados, pesquisas futuras podem evoluir em diferentes frentes. A partir dos conjuntos de dados que produzimos, novos estudos podem ser realizados sobre a previsão ou classificação dos mais diversos padrões em ITS. Relacionado à exterocepção, podem ser reconhecidos buracos, rachaduras, trincas, relevo, entre outros. Relacionados à propriocepção, podem ser reconhecidos eventos de condução como aceleração, frenagem, virando à direita ou esquerda, assim como o nível de segurança de condução.

Em relação aos modelos de percepção veicular produzidos nesta pesquisa, novos estudos podem analisar o comportamento do modelo com dados de diferentes pontos de coleta utilizados em conjunto; realizar experimentos com a permutação dos conjuntos de dados agrupados por fatores de dependência; desenvolver modelos com aprendizado contínuo; experimentar novas características de entrada no domínio da frequência ou tempo-frequência; realizar segmentação das amostras em função do tempo/distância utilizando \textit{downsampling} quando existir mais amostras que o necessário para a entrada da DNN, ou \textit{padding} quando a janela for menor que a necessária; utilizar nossos modelos de DNN como GT para treinar modelos baseados em visão computacional; etc. Novas coletas também podem ser produzidas para investigar outros tipos de superfície, como areia, lajotas sextavadas, paver, etc., assim como novas aplicações, tal como avaliação de piso de fábrica na Indústria 4.0.

