\chapter{Conclusões e Discussões}
\label{cap:conclusoes_discussoes}

% ROAD SURFACE TYPE ARTICLE
% With the ITS development, some applications have become increasily present in daily life, such as ADAS and, soon, autonomous vehicles. However, these systems need many sources of situational data to perform their purpose well. Among the various information, the recognition of surface type is one of the most important, and it can be used in the most varied applications to support human or machine decision making. However, for a wide application, it is necessary to develop a recognition model that uses safe technologies, and that is able to understand different contexts through which a vehicle can travel.

% Based on the premise above, in this work we developed models for road surface type classification using inertial sensor signals, which by their passive approach are safe, non-polluting and generally low cost. Through the literature analysis, there are few related studies which were carried out with low accuracy values, without considering contextual variations. As a way to analyze and validate the model for application in real-world scenarios, we list the dependency factors that impact in the learning and generalization of the pattern recognition model, and aggregate them. Initially, we performed several data collections to obtain variations of these factors. After pre-processing these data, they were interpreted using Classical Machine Learning and Deep Learning techniques to identify which one was the most appropriate.

% Among the Classical Machine Learning techniques, we apply KMC, SVM and KNN that obtained low accuracy values in the validation, 60.42\%, 72.68\% and 74.79\%, respectively. In the application of these methods, we observed the difficulty of extracting high-level features to well represent the data classes, even when applied statistical methods which are widely adopted in the feature extraction of vibration-based data. In the application of Deep Learning approach, all techniques were able to extract complex features from the data, so in all the developed DNN models, accuracy values were greater than those obtained through the best developed model based on classical techniques. LSTM-based, CNN-based and CNN-LSTM-based models were analyzed, which obtained the best validation accuracy result of 92.73\%, 93.17\% and 92.77\%, respectively. In a general analysis, we consider the CNN-based model the best due to the f1-score and accuracy value. The model classified segments of dirt road with f1-score value of 90.85\%, cobblestone road with 85.84\% and asphalt road with 98.96\%. With these results, we notice that all the data classes were properly handled by the model, and it is not necessary to apply data class balancing. In this work, we did not compare our models with those related in the state-of-the-art due to the lack of detail on the methodology of related studies.

% With the public availability of the datasets that we produce, new studies can be carried out on the recognition and classification of the most diverse patterns in ITS. Related to the exteroception, can be recognized potholes, speed bumps, surface quality, among others. Related to proprioception,  can be recognized driving events such as acceleration, braking, turning right or left, as well as driving safety level. In relation to the road surface type classification, new researches can perform experiments with the permutation of datasets grouped by dependency factors, analyze the other data collection points, such as near and above the suspension, and on the dashboard, both present in the collected dataset. Deep Learning layers with different approaches can also be analyzed, such as the Gated Recurrent Unit (GRU) and Convolutional LSTM (ConvLSTM).

% MULTI CONTEXTUAL ROAD

% Through the development of numerous applications in ITS, the need to have a diversity of situational information from different sources has emerged to make these systems more robust. Among the information produced, the classification of the road surface type is one of the most important and can be used in the most varied applications of ITS. However, for this information to be widely used, it is necessary that the source of the raw data is secure and that the model that interprets them to generate the classification is reliable.

% Based on the above premise, in this work we developed classification models based on Deep Learning to process data from inertial sensors. These sensors, given their passive approach, are safe, non-polluting, and low cost, constituting an interesting alternative for large-scale use. However, through the literature analysis, we observed that the models proposed so far for this purpose have a low values of evaluation metrics, without considering the contextual variations to which the model must be submitted. In order to develop and validate a reliable model for application in real-world scenarios, we raise the dependency factors that affect the data of these sensors and, consequently, the process of learning and generalizing the models. These factors were grouped into four properties, sensory, vehicular, environmental, and driving properties.

% Through the establishment of dependency properties, we perform several data collections to obtain variations of these factors. The variations included data collected in three different placements of the vehicle, in three different vehicles, driven by three drivers, traveling on three surface types which presented variations of obstacles and anomalies. These data were then pre-processed, grouping them in specific experiments to analyze the impact of the collection point, the analysis domain, the input features, the size of the data window, and the ability to generalize to unknown contexts. All of these experiments were performed on three different DNN models, being LSTM-based, GRU-based, and CNN-based.

% The models that we developed aimed to classify the data between dirt road, cobblestone road, and asphalt road segments. With the experiments carried out, we observed that the data window has little impact on the final result. Concerning the analysis domain, the use of features in the time domain obtained results similar to features in the frequency domain, but with a lower computational cost. The aggregation of composite acceleration features only made the model more expensive, without increasing the accuracy values. Regarding the collection point and the ability to generalize to unknown contexts, all networks obtained good accuracy values at all data collection points and in all experiments by contextual variation, denoting the ability of DNNs to understand the relationships between the data, even variations in dependency properties. In a macro analysis, we chose the CNN-based DNN as the best model, which has an average validation accuracy between data collection points of 92.70\%, with average f1-score of 90.32\% for the dirt road, 85.34\% for cobblestone road, and 98.66\% for asphalt road. We observed that the results obtained are superior to previous studies, adding that in this study we analyzed these results in various contexts, with variations in the dependency properties.

% With the availability of the datasets we produce, further studies can be carried out to improve the recognition and classification of road surface type. These improvements can be made through the tuning of the hyperparameters of the proposed models, as well as experiments with hybrid DNN, as is the case of LSTM-CNN, GRU-CNN, and ConvLSTM. Regarding the analysis domains, other features can be applied through the time, frequency, and time-frequency domain. Other types of situational information can also be produced using PVS datasets, such as pothole and speed bump detection.

% SPEED BUMP
% Transport systems have been established throughout history as an important condition for human development. Due to their vital importance, ITS emerged to make transport networks more efficient and safer. ITS applications are already present in daily life, such as ADAS, and tend to be more common soon, like autonomous vehicles. However, for these applications to operate reliably, it is necessary to use secure technologies to collect raw data; the diversification and redundancy of these data sources; the development of Artificial Intelligence models that operate correctly in real contexts; and the generation of a wide variety of situational information about the traffic environment and its participants. 

% Among the situational information of great importance for ITS is the identification of obstacles on the road, such as speed bumps. Through our SLR we identified that the related studies use inertial sensors or cameras as a source of the raw data used to recognize speed bumps. Due to their passive approach, both sensors are considered safer, non-polluting, and generally low cost. However, while camera-based solutions are mature, with analysis in different contextual conditions, the same is not happen for solutions based on inertial sensors. Therefore, we proposed in this study the development of speed bump detection models based on signals from inertial sensors that learn in different contextual conditions. 

% In order to develop and validate a reliable model for application in real-world scenarios, we identify dependency factors that affect the data of inertial sensors and, consequently, the process of learning and generalizing the models. These factors were grouped into four properties: sensory, vehicular, environmental, and driving properties. Through the establishment of dependency properties, we perform several data collections to obtain variations of these factors. The variations included data collected in three different placements of the vehicle, in three different vehicles, driven by three drivers, traveling on three road surface types which presented variations of obstacles and anomalies. These data were then pre-processed, grouping them in specific experiments to analyze the impact of the collection placement, the size of the data window, and the ability to generalize to unknown contexts, as an unknown vehicle, driver, or environment. All of these experiments were performed on five different DNN models, being LSTM-based, GRU-based, CNN-based, CNN-LSTM-based, and ConvLSTM-based. In our analysis, we considered CNN-LSTM-based as the best model, which obtained an average accuracy value of 98.59\%, a precision of 95.99\%, a recall of 98.17\%, and an f1-score of 96.97\%, showing a reliable model for application in different and real contexts. 

% With the availability of all the material in this study, such as datasets, source-codes, and trained models, future research may evolve on different fronts. Using the datasets, new studies can be carried out on the forecasting or classification of the most diverse patterns in ITS. Related to the exteroception, can be recognized potholes, cracks, surface quality, among others. Related to proprioception, can be recognized driving events such as acceleration, braking, turning right or left, as well as driving safety level. Regarding the detection of speed bumps, new studies can analyze the behavior of the model with data from different collection points used together; can perform experiments with the permutation of datasets grouped by dependency factors; they can give more details about the recognized speed bumps, such as height, width, or integrate to our Deep Learning model that classifies the road surface type \cite{Menegazzo2020_1}, detailing the type of pavement of the speed bump; develop a model with continuous learning; experiment with new input features in the frequency or time-frequency domain; perform segmentation of samples according to time/distance using downsampling when there are more samples than necessary for the entry of the DNN, or padding when the window is smaller than necessary; use our DNN model as a GT to train models based on computer vision; etc.

\section{Trabalhos Futuros}