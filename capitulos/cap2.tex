\chapter{Metodologia}
\label{cap:metodologia}

Este trabalho utiliza do método científico hipotético-dedutivo proposto pelo filósofo austríaco Karl Popper \cite{Popper2002}. Neste método, partindo de hipóteses estabelecidas, são eleitas para experimentação aquelas que possuem certa viabilidade para responder um determinado problema de natureza científica. Após essa eleição, busca-se o falseamento das hipóteses, objetivando comprovar sua sustentabilidade. O método termina com a comprovação das hipóteses; ou com a construção de novas, caso as atuais sejam refutadas \cite{Bonat2009}.

A metodologia deste trabalho se baseia nas quatro práticas operacionais que consagram o método científico: a identificação de um problema, a indicação de uma hipótese, a coleta de dados e a análise da resposta \cite{Bonat2009}. Desta forma, partindo da formulação do problema de pesquisa, foi realizado o levantamento do estado da arte através de uma revisão sistemática da literatura. Com este levantamento, o problema foi refinado para criação da seguinte hipótese: 

\begin{description}
\item{\textbf{Hipótese:}} Através de um conjunto de dados que bem represente a diversidade contextual envolvida na aplicação dos sensores inerciais em ITS para geração de percepção veicular, ou seja, que os dados contenham variações contextuais significativas em relação aos fatores de dependência da solução, é possível construir modelos de Inteligência Artificial capazes de aprender as relações e as influências dos fatores de dependência nos sinais dos sensores, possibilitando a generalização de seu aprendizado para cenários desconhecidos de forma confiável.
\end{description}

Partindo desta hipótese, foi realizada a coleta de dados respeitando a metodologia necessária para efetuar o estudo de validação. Para este estudo, foi definido um \textit{design} experimental no qual se pudesse testar a veracidade da hipótese levantada. Por fim, de forma a mensurar os resultados dos experimentos para comprovar ou refutar a hipótese, métricas específicas de avaliação dos modelos foram adotadas. Cada uma destas etapas é detalhada nas próximas seções.

\section{Levantamento do Estado da Arte}

Esta etapa tem como objetivo o levantamento do estado da arte acerca da aplicação de sensores inerciais em ITS para geração de percepção veicular. Desta forma, foi conduzida uma Revisão Sistemática da Literatura (RSL) baseada nos procedimentos descritos em \cite{kitchenham2004,kitchenham2009,biolchini2005}. A RSL foi dividida em três fases: definição, execução e análise. Na primeira, foi delimitado o escopo de busca, definidas as perguntas de pesquisa, especificadas as bases científicas utilizadas e suas \textit{strings} de busca, além dos critérios de inclusão e exclusão dos estudos primários. Na segunda fase, foi executada a busca e selecionados os trabalhos que aderiram aos critérios de seleção definidos na fase anterior. Por fim, na terceira fase, cada estudo foi analisado na íntegra, com extração e sumarização das informações de interesse.

A revisão produzida buscou responder as perguntas de pesquisa, mapeando aspectos importantes das etapas de coleta de dados, pré-processamento e processamento. Desta forma, relacionado a coleta de dados, identificou-se o contexto no qual a amostragem de sinais ocorreu, as plataformas de \textit{hardware} utilizadas, os sensores empregados e suas configurações tais como taxa de amostragem e quadros referenciais, além da forma de colocação e posicionamento dos dispositivos na infraestrutura veicular. Em relação à etapa de pré-processamento, foram identificados os métodos e técnicas aplicadas para organizar e transformar os dados antes de serem utilizados como entrada dos modelos de IA, tais como reorientação de eixos, filtragem e normalização de sinais, extração de características, segmentação de dados, etc. Por fim, na etapa de processamento foram identificadas as técnicas utilizadas para reconhecimento dos padrões. Através de uma análise ampla sobre as três etapas discorridas, foram levantados também os fatores de dependência que impactam na adaptabilidade da solução, os tipos de percepção possíveis de serem produzidos, além das áreas de aplicação.

\section{Coleta de Dados}

Através de nossa RSL, não encontramos nenhum conjunto de dados público de sensores inerciais embarcados em carros que permitisse a análise proposta. Sendo assim, produzimos nove conjunto de dados através de diversos sensores de abordagem passiva. Foram utilizadas duas redes de sensores, sendo cada uma delas constituídas por um \textit{Single-Board Computer} (SBC) Raspberry Pi e três placas MPU-9250, cada uma equipada com um acelerômetro, um giroscópio, um magnetômetro e um sensor de temperatura. Também foi empregada uma fonte de \textit{Global Positioning System} (GPS), com produção de dados de localização e velocidade, além de uma câmera para captura de vídeo ambiente.

A coleta de dados foi realizada de forma a amostrar dados em variações contextuais dos fatores de dependência. Estes fatores são considerados \emph{propriedades sensoriais}, como faixa de medição, resolução, referenciais de captação e análise; \emph{propriedades veiculares}, como sistema de suspensão e mecânica veicular; \emph{propriedades de condução}, como velocidade aplicada ao veículo e estilo de condução; e \emph{propriedades ambientais}, como alterações de condição de conservação, presença de irregularidades e obstáculos, etc. Sendo assim, os sensores foram distribuídos no veículo através de posicionamento controlado, onde a colocação dos módulos foi realizada de forma que os três eixos do sistema de coordenadas do sensor ficaram alinhados com os do veículo, sendo tanto referencial de coleta como de análise. Também foram definidas configurações adequadas para não saturar as leituras dos sensores, tais como \textit{Full Scale Range} (FSR) e resolução, além da taxa de amostragem, satisfazendo a dependência da primeira propriedade.

Para se adequar a metodologia de validação proposta, as três propriedades restantes necessitam apresentar variações contextuais. Sendo assim, em adição a amostragem de velocidade (propriedade de condução) e colocação dos sensores em três diferentes pontos da infraestrutura veicular (propriedade veicular), nós realizamos as coletas de dados com o conjunto de sensores aplicados a três diferentes veículos (propriedade veicular), com três diferentes motoristas variando a velocidade de 0 \emph{km/h} até 91,98 \emph{km/h} (propriedade de condução), cobrindo três cenários distintos (propriedade ambiental). Cada cenário conta com três tipos de superfícies, dentre pavimentadas (paralelepípedo e asfalto) e não pavimentadas (terra), além de variações ambientais gerais, tais como presença de lombadas, buracos, variação na conservação dos pavimentos, aclives e declives, etc.

Com os dados coletados, foram criadas as classes de dados a serem utilizadas como \textit{Ground Truth} (GT) dos experimentos. Para a classificação de tipo de superfície de pista, foram criadas as classes terra, asfalto e paralelepípedo. Para a classificação de qualidade de pista, criou-se os níveis ruim, regular e bom. Para a detecção de lombadas, criou-se as classes com e sem lombada. As classes de tipo de superfície e presença de lombadas remetem a elementos objetivos, onde visualmente pode-se estabelecer seu início e fim. Logo, estas duas contam com GT de anotação humana \cite{Krig2014}. Já a qualidade de superfície torna-se subjetiva se realizada com este tipo de GT. Sendo assim, desenvolvemos um método de extração de características de vibração que envolvem as três últimas propriedades de dependência. Logo, as classes referentes aos níveis de qualidade contam com GT de anotação automatizada por máquina \cite{Krig2014}.

\section{Experimentos e Validação}

Os modelos de IA desenvolvidos foram experimentados de forma a verificar a técnica mais adequada para o reconhecimento dos padrões, dentre abordagens de \textit{Machine Learning} clássico e \textit{Deep Learning}, além de avaliar aspectos diversos, como janela de dados ideal, domínio de análise, ponto de coleta, etc. Contudo, para validar a hipótese deste estudo, nós desenvolvemos um \textit{design} experimental com variação das propriedades de dependência veiculares, de condução e ambientais. Convém mencionar que a propriedade sensorial é a única que não depende do aprendizado do modelo, mas sim da metodologia de coleta de dados. Sendo assim, enquanto as demais influenciam o valor quantificado nas medições, podendo aumentar ou diminuir a amplitude, as propriedades sensoriais influenciam a corretude do valor, onde suas configurações e posicionamento podem resultar na saturação dos dados ou referencial incorreto.
 
Os nove conjuntos produzidos se relacionam proporcionalmente com as três propriedades de dependência de interesse nessa etapa, através da combinação de três veículos, três motoristas e três cenários/ambientes. Sendo assim, a metodologia de validação consiste em três experimentos, onde em cada um deles existe duas variações de propriedade para treinamento, e a terceira variação aparece somente em validação. Desta forma, analisamos o a generalização de aprendizado que o modelo obteve através de dois veículos, quando submetido a um terceiro. O mesmo ocorre com motoristas e ambientes. Sendo assim, conseguimos avaliar o comportamento do modelo quando submetido a um carro, motorista ou ambiente desconhecido. Para mensurar os resultados, utilizamos das métricas de acurácia, precisão, \textit{recall} e \textit{f1-score}.