\begin{abstract}
Transport systems have been established throughout history as one of the main constraints to human development. With the advancement of computational technologies, intelligent transport systems emerged, in which sensors are used in the transport infrastructure and its participants, in order to generate raw data that, processed by AI models, generate situational information about the modal. In this context, several technologies with an active and passive approach have been developed. Among the passive technologies, there are those based on vibration, performed through inertial sensors, which can generate information in the form of vehicle perceptions, in a safe, non-polluting and low-cost way. However, in contrast to areas such as computer vision, inertial sensing has been little explored, where the solutions proposed in the literature are not adaptable for wide application in real-world scenarios, usually configuring themselves as proof of concept through simple models. In this way, given the contextual diversity in which the solution can be submitted, there are several dependency factors that interfere and influence the values of the signals sampled with these sensors, so that the adaptability of the solution to these factors is an essential requirement to provide reliability and, in turn, enable a wide application. With this objective in mind, in this work proposes the development of vehicle perception models based on inertial sensor signals, capable of operating reliably in contextual variations related to dependency factors: different vehicles, driving styles and environments. In this work focused on the development of perceptions of road surface type, road surface quality, and speed bump detection. For the development and validation of the models, nine datasets with contextual variations were collected, using three vehicle models, with three different drivers, in three different environments, in which there are three different surface types, in addition to variations in the conservation state and the presence of obstacles and irregularities. The collected data were used in experiments to evaluate aspects such as the influence of the vehicle's data collection point, the analysis domain, the model's input features and the data window size. Subsequently, the models' ability to generalize their learning to unknown contexts was evaluated, i.e., their behavior when applied to data sampled in an unknown vehicle, driver or environment, thus analyzing their adaptability. The experiments were carried out with models based on classical and deep machine learning, where the best model for road surface type classification was a CNN network, which classified dirt, cobblestone and asphalt segments with an average accuracy of 92.70\%; the best model for road quality classification was a CNN network, which classified segments into good, fair and poor levels with an average accuracy of 93.52\%; and the best model for speed bump recognition was a hybrid CNN-LSTM network, which detected speed bumps with an average accuracy of 98.59\%.
 
 \vspace{\baselineskip} 
  \textbf{Palavras-chave:} 
  Road Surface Type Classification.
  Road Surface Quality Classification.
  Speed Bump Detection.
  Inertial Sensors. 
  Deep Learning.
\end{abstract}

%  Enlish version of the plain ``resumo'' above. Done with environment
%   \texttt{abstract}. Hyphenization is automatically changed to english.

%   \vspace{\baselineskip} 
%   \textbf{Keywords:} Keyword. Another Compound Keyword. Bla.