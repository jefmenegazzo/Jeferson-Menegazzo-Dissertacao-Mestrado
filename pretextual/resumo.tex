%  Aqui deve ser inserido um resumo de 150 a 500 palavras (limitação de tamanho dada pela BU). A linguagem deve ser português e a hifenização já foi alterada. O resumo em português deve preceder o resumo em inglês, mesmo que o trabalho seja escrito em inglês. A BU também diz que deve ser usada a voz ativa e o discurso deve ser na 3ª pessoa. A estrutura do resumo pode ser similar a estrutura usada em artigos: Contexto -- Problema -- Estado da arte -- Solução proposta  -- Resultados.

\begin{resumo}[Resumo]
    Os sistemas de transporte se estabeleceram ao longo da história como um dos principais condicionantes ao desenvolvimento humano. Com o avanço das tecnologias computacionais, surgiram os sistemas de transporte inteligentes, nos quais sensores são empregados na infraestrutura de transporte e seus participantes, de forma a gerar dados brutos que, processados por modelos de IA, geram informações situacionais acerca do modal. Neste contexto, foram desenvolvidas diversas tecnologias de abordagem ativa e passiva. Dentre as tecnologias passivas, existem as baseadas em vibração, realizadas através de sensores inerciais, as quais podem gerar informações na forma de percepções veiculares, de forma segura, não poluente e de baixo custo. Entretanto, ao contrário de áreas como a visão computacional, o sensoriamento inercial tem sido pouco explorado, onde as soluções propostas na literatura não são adaptáveis para ampla aplicação em cenários do mundo real, se configurando normalmente como prova de conceito através modelos simples. Neste sentido, dada a diversidade contextual na qual a solução pode ser submetida, existem diversos fatores de dependência que interferem e influenciam os valores dos sinais amostrados com estes sensores, de forma que a adaptabilidade da solução a estes fatores se mostra um requisito essencial para prover confiabilidade e, por sua vez, possibilitar uma ampla aplicação. Com este objetivo, neste trabalho foi proposto o desenvolvimento de modelos de percepções veiculares baseados em sinais de sensores inerciais, capazes de operar de forma confiável em variações contextuais relacionados aos fatores de dependência: diferentes veículos, estilos de condução e ambientes. Neste trabalho focou-se no desenvolvimento das percepções de tipo de superfície de pista, qualidade de superfície de pista, e detecção de lombadas. Para o desenvolvimento e validação dos modelos, foram coletados nove conjuntos de dados com variações contextuais, utilizando três modelos de veículos, com três motoristas, em três ambientes distintos, nos quais existem três tipos de superfície, além de variações no estado de conservação e a presença de obstáculos e irregularidades. Os dados coletados foram utilizados em experimentos para avaliar aspectos como a influência do ponto de coleta de dados do veículo, o domínio de análise, as características de entrada do modelo e a janela de dados. Posteriormente, foi avaliada a capacidade de generalização do aprendizado dos modelos para contextos desconhecidos, ou seja, seu comportamento quando aplicado a dados amostrados em um veículo, motorista ou ambiente desconhecido, analisando assim sua adaptabilidade. Os experimentos foram realizados com modelos baseados em aprendizado de máquina clássico e \textit{deep learning}, onde o melhor modelo para classificação de tipo de superfície foi uma rede CNN, a qual classificou segmentos de terra, paralelepípedo e asfalto com acurácia média de 92,70\%; o melhor modelo para classificação de qualidade foi uma rede CNN, a qual classificou segmentos nos níveis bom, regular e ruim com acurácia média de 93,52\%; e o melhor modelo para reconhecimento de lombadas foi uma rede híbrida CNN-LSTM, a qual detectou lombadas com acurácia média de 98,59\%.
    
    % Atenção! a BU exige separação através de ponto (.). Ela recomanda de 3 a 5 keywords
  \vspace{\baselineskip} 
  \textbf{Palavras-chave:} 
  Classificação de Tipo de Superfície de Pista.
  Classificação de Qualidade de Superfície de Pista.
  Detecção de Lombadas.
  Sensores Inerciais. 
  Aprendizado Profundo.
  
\end{resumo}
